\documentclass{article}
\usepackage[spanish]{babel} %Definir idioma español
\usepackage[utf8]{inputenc} %Codificacion utf-8
\usepackage{amssymb, amsmath, amsbsy, wasysym}
\usepackage{multirow} % para tablas
\usepackage{graphicx}
\author{Emmanuel Peto Gutiérrez}
\title{Tarea 1}
\begin{document}
\maketitle

El programa está hecho en \textit{Haskell}, así que para compilar y ejecutar se teclea lo siguiente en la carpeta src:

\begin{itemize}
\item ghc Saturacion.hs
\item ./Saturacion
\end{itemize}

\section{Forma normal conjuntiva}

Para pasar cualquier fórmula a forma normal conjuntiva primero se eliminaron equivalencias ($\leftrightarrow$) e implicaciones ($\rightarrow$). Luego se colocaron las negaciones solamente en atómicas para pasar a forma normal negativa. Finalmente de FNN se transformó a FNC usando reglas de distribución.

\section{Conjunto de cláusulas}

Para transformar una fórmula en FNC a su forma de conjuntos, primero se hizo una función que dada una cláusula devuelve una lista de sus literales. Después se usó esta función para hacer una lista de cláusulas.

\section{Saturación}

Se realizó una función que encuentra literales complementarias entre dos cláusulas; si no tienen literales complementarias devuelve \textbf{Nothing}. La función \textit{resBin} realiza resolución binaria de dos cláusulas dada una literal \textbf{l}. Después se hizo una función que hace resolución binaria de todas las cláusulas en un conjunto y la une con el conjunto original. Finalmente se aplica el algoritmo de saturación de forma recursiva y este se detiene si el conjunto de cláusulas no cambia o aparece el conjunto vacío en algún punto.

\section{Saturación con simplificación}

Para simplificar un conjunto de cláusulas se realizan los siguientes pasos:

\begin{itemize}
\item Se eliminan las literales repetidas de cada cláusula por la propiedad de idempotencia.
\item Se eliminan las cláusulas que son tautología. Es decir, las que tienen literales complementarias o que tienen \textbf{T}.
\item Se eliminan las cláusulas que son subsumidas por otras. Para esto se creó una función que determina si una cláusula C es superconjunto de algún elmento en un conjunto de cláusulas Cl; si lo es se elimina C.
\end{itemize}

Una vez que se tiene la función de simplificación, se concatena la simplificación con la resolución y se aplica el algoritmo de saturación.

\end{document}
