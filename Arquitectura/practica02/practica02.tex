\documentclass{article}
\usepackage[T1]{fontenc}
\usepackage[utf8]{inputenc}
\usepackage[spanish]{babel}
\usepackage{amssymb, amsmath, amsbsy} % simbolitos
\usepackage{multirow} % para tablas
\usepackage{enumerate}
\title{Práctica 2 \\ Organización y arquitectura de computadoras}
\author{Peto Gutiérrez Emmanuel}
\begin {document}
\maketitle

Explicación de cómo resolví los ejercicios:

\section{Implicación}

Considerando que la implicación es $P \rightarrow Q$, sabemos que cuando P es 0 el resultado va a ser 1. Por lo tanto, usé un transistor tipo P que deja pasar la corriente de 1 (usando "Power") cuando P es 0 y eso resuelve 2 casos. Para resolver los otros 2 (cuando I(P)=1) noté que el resultado es igual a I(Q), es decir, si I(Q)=0 la implicación es 0 y si I(Q)=1 la implicación es 1. Entonces usé el transistor tipo N para dejar pasar la corriente de Q cuando I(P)=1.

\section{Primo}

Se resolvió usando una tabla de verdad y luego reduciendo con un mapa de Karnaugh. El número es de la forma $X=X_{3}X_{2}X_{3}$ donde cada $X_{i}$ es un dígito.

\begin{table}[htbp]
\begin{center}
\begin{tabular}{|l|l|l|l|}
\hline
\textbf{$X_{3}$} & \textbf{$X_{2}$} & \textbf{$X_{1}$} & \textbf{P} \\ \hline
0 & 0 & 0 & 0 \\ \hline
0 & 0 & 1 & 0 \\ \hline
0 & 1 & 0 & 1 \\ \hline
0 & 1 & 1 & 1 \\ \hline
1 & 0 & 0 & 0 \\ \hline
1 & 0 & 1 & 1 \\ \hline
1 & 1 & 0 & 0 \\ \hline
1 & 1 & 1 & 1 \\ \hline
\end{tabular}
\end{center}
\end{table}

\begin{table}[htbp]
\begin{center}
\begin{tabular}{|l|l|l|l|l|}
\hline
$X_{1} \backslash X_{3}X_{2}$ & \textbf{00} & \textbf{01} & \textbf{11} & \textbf{10} \\ \hline
\textbf{0} & 0 & 1 & 0 & 0 \\ \hline
\textbf{1} & 0 & 1 & 1 & 1 \\ \hline
\end{tabular}
\end{center}
\end{table}

Obtenemos la función: $\overline{X_{3}}X_{2}+X_{2}X_{1}+X_{3}X_{1}$ con lo cual podemos hacer el circuito.

\section{Menor o igual}

Para hacer la igualdad solo verifiqué que cada dígito fuera igual con $x \leftrightarrow y$, que es equivalente a $(x \wedge y) \vee (\lnot x \wedge \lnot y)$.
Para verificar que $X=X_{2}X_{1}$ es menor que $Y=Y_{2}Y_{1}$ se comprueba que: \\
$X_{2} < Y_{2}$ o bien que $X_{2} = Y_{2}$ \& $X_{1} < Y_{1}$. Para eso usamos la equivalencia lógica:
$p<q \equiv \lnot p \wedge q$.

\section{Elevador}

Supongamos que $X$ es la posición actual del elevador y $Y$ el piso al que se quiere ir. Para saber la dirección solo usamos la función menor o igual, definida anteriormente, donde 1 es arriba y 0 es abajo.

Para saber cuántos pisos se moverá hacemos una función para sacar el valor absoluto de la diferencia entre $X$ y $Y$, es decir $|X-Y|$. Para ello hice todas las restas posibles y noté que se podía obtener cada dígito del resultado mediante un XOR a los dígitos de las entradas, excepto en los casos $X=10$ y $Y=01$ o viceversa. Para estas excepcions solo tengo que invertir un dígito y después aplicar el XOR. Es decir, si tenemos $X=01$ y $Y=10$ invierto el dígito más significativo de $Y$, entonces queda: $Y=00$. Si tenemos $X=10$ y $Y=01$ invierto el dígito más significativo de la $X$ y da $X=00$.

\section{Preguntas}

\begin{enumerate}[1.]
	\item ¿Cuál es el procedimiento a seguir para desarrollar un circuito que resuelva un problema que involucre lógica computacional?
	\begin{itemize}
        \item Hacer una tabla de verdad con todos los posibles valores de entrada del problema.
        \item En la columna que representa la función de conmutación colocar 1 donde necesitemos que la función de 1.
        \item Obtener mintérminos multiplicando las variables en los renglones donde la función tiene 1. Si la variable tiene 0 en ese renglón, la negamos.
        \item Sumar cada uno de los mintérminos para crear la función.
        \item Si se puede, reducir el tamaño de la función usando mapas de Karnaugh o propiedades del álgebra booleana.
	\end{itemize}
	
	\item Si una función de conmutación se evalúa a más ceros que unos ¿es conveniente usar mintérminos o maxtérminos? ¿En el caso de que evalúe a más ceros que unos?
	\begin{itemize}
		\item Si se tienen más ceros que unos es mejor usar maxtérminos, pues en la expresión de la función solo aparecerán los términos de los renglones que tienen 0.
		\item Si se tienen más unos que ceros es más conveniente usar mintérminnos, además se tiene la posibilidad de reducir la expresión.
	\end{itemize}
	
	\item Analizando el trabajo realizado, ¿cuáles son los inconvenientes de desarrollar circuitos de forma manual?
	\begin{itemize}
		 \item Si usamos tablas de verdad para crear un circuito con n variables, tendremos $2^{n}$ renglones, por lo que el problema crece exponencialmente.
		 \item Se enciman los cables y eso puede provocar confusión.
	\end{itemize}
\end{enumerate}

\end {document}