\documentclass{article}
\usepackage[T1]{fontenc}
\usepackage[utf8]{inputenc}
\usepackage[spanish]{babel}
\usepackage{amssymb, amsmath, amsbsy} % simbolitos
\usepackage{multirow} % para tablas
\usepackage{enumerate}
\title{Práctica 7 \\ Organización y arquitectura de computadoras}
\author{Peto Gutiérrez Emmanuel}
\begin {document}
\maketitle
\begin{enumerate}[1.]
\item \textbf{Considerando los componentes básicos de una computadora moderna, además de las llamadas al sistema descritas en la práctica, ¿qué otras resultarían necesarias?}
	\begin{itemize}
	\item Creat para crear un archivo.
	\item System para ejecutar un comando del shell.
	\item Fork para crear un proceso.
	\end{itemize}
\item \textbf{¿Qué comandos son básicos para un intérprete de comandos de un sistema operativo?}
	\begin{itemize}
	\item ls: Para listar los archivos y carpetas en el directorio.
	\item cd: Para cambiar de directorio.
	\item mkdir: Para crear una carpeta.
	\item cp: Para copiar un archivo o carpeta.
	\item mv: Para mover(cortar) un archivo o carpeta.
	\item rm: Para remover un archivo o carpeta.
	\end{itemize}
\item \textbf{Investiga y describe con tus propias palabras ¿cómo resuelve una llamada al sistema el sistema operativo?}
	\begin{itemize}
	\item Los lenguajes de programación proveen una interfaz de llamadas al sistema, que se comunican con las llamadas del sistema operativo. Un número está asociado a cada llamada al sistema y la interfaz de llamadas al sistema tiene una tabla indexada de acuerdo a estos números. La interfaz invoca la llamada del sistema operativo y devuelve un valor si es necesario.
	\end{itemize}
\end{enumerate}
\end {document}