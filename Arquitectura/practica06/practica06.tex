\documentclass{article}
\usepackage[T1]{fontenc}
\usepackage[utf8]{inputenc}
\usepackage[spanish]{babel}
\usepackage{amssymb, amsmath, amsbsy} % simbolitos
\usepackage{multirow} % para tablas
\usepackage{enumerate}
\title{Práctica 6 \\ Organización y arquitectura de computadoras}
\author{Peto Gutiérrez Emmanuel}
\begin {document}
\maketitle

\begin{enumerate}[1.]
\item \textbf{¿Qué utilidad tiene el registro \$fp ? ¿Se puede prescindir de el?}
	\begin{itemize}
	\item Sirve para delimitar el tamaño del marco y no salirnos de este espacio para guardar cosas en la pila.
	\item Es posible no utilizar este registro si recordamos cuánto espacio reservamos en la pila.
	\end{itemize}

\item \textbf{Definimos como subrutina nodo a una subrutina que realiza una o más invocaciones a otras subrutinas y como subrutina hoja a una subrutina que no realiza llamadas a otras subrutinas.}
	\begin{enumerate}[a)]
	\item ¿Cuál es el tamaño mínimo que puede tener un marco para una subrutina nodo? ¿Bajo qué condiciones ocurre? \\
	
	El tamaño mínimo son 24 bytes; 16 para los argumentos, 4 para el \$fp y 4 para el \$ra. Ocurre si no se van a modificar registros \$sn y si no se van a utilizar registros \$tn después de la llamada.
	\\
	\item ¿Cuál es el tamaño mínimo que puede tener un marco para una subrutina hoja? ¿Bajo qué condiciones ocurre? \\
	
	El tamaño mínimo son 4 bytes, para guardar el \$fp del marco anterior. Ocurre si no se van a modificar los registros \$sn ni los registros de argumentos \$an.
	\end{enumerate}
\item \textbf{Considera el siguiente pseudocódigo. En donde a[5] es un arreglo de tamaño 5 y "..."  son otras acciones que realiza la rutina, además, supón que en la funcion B se realizan cambios en los registros \$s0, \$s1 y \$s2 . Bosqueja la pila de marcos después del preámbulo de la funcion B.}

funcion\_A(a,b)

\hspace{0.5cm}a[5]

\hspace{0.5cm}...

\hspace{0.5cm}funcion\_B(a,b,arreglo[0],arreglo[1],arreglo[2])

\hspace{0.5cm}...
\end{enumerate}

\begin{table}[htbp]
\begin{center}
\begin{tabular}{|c|}
\hline
arreglo[4] \\ \hline
arreglo[3] \\ \hline
arreglo[2] \\ \hline
arreglo[1] \\ \hline
arreglo[0] \\ \hline
\$fp \\ \hline
\$ra \\ \hline
0 \\ \hline
0 \\ \hline
\$a0 \\ \hline
\$a1 \\
\hline \hline
\$s2 \\ \hline
\$s1 \\ \hline
\$s0 \\ \hline
argumento 5 \\ \hline
\$a3 \\ \hline
\$a2 \\ \hline
\$a1 \\ \hline
\$a0 \\ \hline
\end{tabular}
\end{center}
\end{table}
\end {document}