\documentclass{article}
\usepackage[T1]{fontenc}
\usepackage[utf8]{inputenc}
\usepackage[spanish]{babel}
\usepackage{amssymb, amsmath, amsbsy} % simbolitos
\usepackage{multirow} % para tablas
\usepackage{enumerate}
\title{Práctica 8 \\ Organización y arquitectura de computadoras}
\author{Peto Gutiérrez Emmanuel}
\begin {document}
\maketitle
\begin{enumerate}[1.]
\item En un procesador, ¿qué es el \textit{modo supervisor}? ¿qué funciones tiene? ¿cómo se implementa?
	\begin{itemize}
	\item Es un modo de operación que tiene acceso a de instrucciones privilegiadas del procesador.
	\item Algunas funciones de este modo son: control de entrada y salida, manejo del timer y manejo de interrupciones.
	\item Se implementa cambiando un bit de modo que está en el hardware. Cambiar el bit de modo es algo que realiza el sistema operativo, pero permite al usuario acceder a algunas instrucciones privilegiadas mediante llamadas al sistema.
	\end{itemize}
\item ¿Cuál es la relación entre una llamada al sistema y una excepción?
	\begin{itemize}
	\item Ambas son señales que se envían al procesador para interrumpir lo que está haciendo y ejecutar un servicio del sistema operativo.
	\end{itemize}
\item ¿Qué es un vector de interrupciones?
	\begin{itemize}
	\item Es una tabla que contiene direcciones a rutinas de servicios de interrupción. Sirve para manejar varias interrupciones.
	\end{itemize}
\end{enumerate}
\end {document}