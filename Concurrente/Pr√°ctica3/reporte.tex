\documentclass{article}
\usepackage[T1]{fontenc}
\usepackage[utf8]{inputenc}
\usepackage[spanish]{babel}
\usepackage{amssymb, amsmath, amsbsy} % simbolitos
\usepackage{multirow} % para tablas
\usepackage{enumerate}
\title{Práctica 3\\Computación concurrente}
\author{Peto Gutiérrez Emmanuel}
\begin {document}
\maketitle
Filósofo \\

Begin

\hspace{0.5cm}while(true)\{

\hspace{1cm}pensar()

\hspace{1cm}wait(palillo\_izquierdo)

\hspace{1cm}wait(palillo\_derecho)

\hspace{1cm}comer()

\hspace{1cm}signal(palillo\_derecho)

\hspace{1cm}signal(palillo\_izquierdo)

\hspace{0.5cm}\}

End \\

El problema no se resuelve porque todos los filósofos toman primero el palillo izquierdo. Si todos terminan de pensar al mismo tiempo, todos tomarán el palillo que tienen a la izquierda al mismo tiempo, después al intentar tomar el derecho se dormirán al llamar el wait(palillo\_derecho), esto es porque todos los palillos que tienen a su derecha ya están ocupados. Esto ocasionará un deadlock. \\

Para resolver el problema se utilizó un semáforo general que le permite la entrada a solo 4 hilos a la sección crítica. Es decir, solo 4 filósofos podrán intentar comer a la vez y el quinto filósofo(el que llegó al final) deberá esperar a que uno de ellos termine de comer para intentar tomar los palillos y eso lo hace libre de deadlock. Antes de que intenten tomar los palillos siempre habrá 1 filósofo dormido, entonces, si todos toman su palillo izquierdo podemos asegurar que el filósofo a la izquierda del dormido va a comer.
\end {document}
